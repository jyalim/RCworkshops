\documentclass[11pt]{article}
\usepackage[margin=1in]{geometry}

\usepackage{amsmath,amsfonts,amssymb,amsthm,latexsym,float}
\usepackage{graphicx}
\usepackage{multicol,bm,mathtools}
\usepackage{listings}
\usepackage{tabularx}
\usepackage{xcolor}
\usepackage[export]{adjustbox}

\usepackage{fontspec}
\setmonofont{SF-Mono}[
   Path= /packages/public/fonts/,
   Extension= .otf,
   BoldFont = *-Bold,
   UprightFont = *-Regular,
   Scale=MatchLowercase,
 ]

\usepackage{soul}

\usepackage{hyperref}
\hypersetup{
  breaklinks=true,
  colorlinks,
  citecolor=blue,
  linkcolor=blue,
  urlcolor=blue,
}

\lstset{ 
  backgroundcolor=\color{gray!10},   % choose the background color; you must add \usepackage{color} or \usepackage{xcolor}; should come as last argument
  basicstyle=\ttfamily\small, % the size of the fonts that are used for the code
% breakatwhitespace=false,           % sets if automatic breaks should only happen at whitespace
% breaklines=true,                   % sets automatic line breaking
% captionpos=b,                      % sets the caption-position to bottom
% commentstyle=\color{mygreen},      % comment style
% deletekeywords={...},              % if you want to delete keywords from the given language
  emph={ssh, git, clone},
	emphstyle=\color{green!60!black},
% escapeinside={\%*}{*)},            % if you want to add LaTeX within your code
  extendedchars=true,                % lets you use non-ASCII characters; for 8-bits encodings only, does not work with UTF-8
% firstnumber=1000,                  % start line enumeration with line 1000
% frame=single,                      % adds a frame around the code
  keepspaces=true,                   % keeps spaces in text, useful for keeping indentation of code (possibly needs columns=flexible)
  keywordstyle=\color{blue},         % keyword style
  language=Bash,                     % the language of the code
  morekeywords={*,mkdir},            % if you want to add more keywords to the set
  numbers=left,                      % where to put the line-numbers; possible values are (none, left, right)
  numbersep=5pt,                     % how far the line-numbers are from the code
  numberstyle=\ttfamily\scriptsize\color{gray},     % the style that is used for the line-numbers
  rulecolor=\color{black},           % if not set, the frame-color may be changed on line-breaks within not-black text (e.g. comments (green here))
  showspaces=false,                  % show spaces everywhere adding particular underscores; it overrides 'showstringspaces'
  showstringspaces=false,            % underline spaces within strings only
  showtabs=false,                    % show tabs within strings adding particular underscores
% stepnumber=2,                      % the step between two line-numbers. If it's 1, each line will be numbered
  stringstyle=\color{mymauve},       % string literal style
  tabsize=2,                         % sets default tabsize to 2 spaces
  title=\lstname                     % show the filename of files included with \lstinputlisting; also try caption instead of title
}

\newcommand{\myurl}[1]{\href{#1}{\color{blue}\setulcolor{blue}\ul{\mbox{#1}}}}

\usepackage{fancyhdr}
\pagestyle{fancy}
\fancyhf{}
\renewcommand{\headrulewidth}{0pt}
\fancyfoot[L]{Quick links for supercomputing and python \href{mailto:jyalim+workshops@email.asu.edu}{jyalim+workshops@email.asu.edu}}
\fancyfoot[R]{Pg.\ \thepage}

\def\ghub{https://github.com/jyalim/RCWorkshops}
\def\gethpc{https://links.asu.edu/getHPC}
\def\docshomepage{https://links.asu.edu/docs}
\def\mamba{https://links.asu.edu/mamba}
\def\whpcclub{https://links.asu.edu/whpc}
\def\whpcform{https://forms.gle/hjwsxKx21LQQYQhX9}

\begin{document}
{\centering\huge\bfseries Supercomputing \& python workshops quick links}

\setcounter{section}{-1}
\section{Getting an account on the supercomputer}

The supercomputer is available for free for Faculty and their students. 
To request an account, fill out the following form: 
\myurl{\gethpc}.
Faculty should list themselves as their own sponsor. 
Students, list your Faculty mentor.
Your account should be created within one business day.

\section{Connecting to the Sol supercomputer}

Once an account is created, you may access the Sol supercomputer from
either your web browser or command-line interface shell. 

\setcounter{subsection}{-1}
\subsection{The VPN}\label{sec:VPN}
If you are off-campus (or on some parts of campus, like Hayden library),
you will need to first connect to the virtual private network (VPN), as
detailed here:
\myurl{https://sslvpn.asu.edu}.

\subsection{Browser connections --- the web portal}\label{sec:ood}

To connect via the browser, go to
\myurl{https://sol.asu.edu}.
You will be prompted to authenticate as if you were signing into your
MyASU.

\subsection{Command-line interface connections --- the shell}\label{sec:shell}

For shell-based connections, run,
\begin{lstlisting}
ssh <asurite>@sol.asu.edu
\end{lstlisting}
where \texttt{<asurite>} should be replaced by your ASURITE, i.e., the
username you sign into MyASU with. If connecting for the first time,
your terminal will ask you to verify the host that you are connecting
to, and then prompt you for your password (which is the password you use
to sign into MyASU).

\section{Launching a Jupyter Server}\label{sec:jupyter}

This is only accessible through the web interface (see
subsection~\ref{sec:ood}). From the web portal's home page, scroll to
the top of the screen to find the navigation bar (in ASU Gold). With
your mouse, select the dropdown ``Interactive Apps.'' From the list,
select, ``Jupyter.'' A new view will populate, which looks like a form
for submitting a Jupyter Lab server.

Set the partition to ``lightwork,'' 
    the QOS to ``public,''
    the CPU core allocation to 1,
    the memory allocation to 2,
    and Jupyter wall time to ``0-2'' (shorthand for 0 days 2 hours).
The other settings should be left in their defaults (for instance, leave
the GPU field blank).

Once the form is filled, click ``Launch,'' and wait for the session to
start on the new page. Once the session is ready, click ``Connect to
Jupyter.''

\section{Accessing the workshop materials}

The workshop materials are available on github, i.e., 
\myurl{\ghub}.
From the Jupyter interface (see section~\ref{sec:jupyter}), scroll to
the bottom of the ``Launcher'' view and click on the ``Terminal'' tile
to open a command-line interface shell.

Once the shell is open, run the following commands:
\begin{lstlisting}
mkdir Desktop
cd Desktop
git clone https://github.com/jyalim/RCWorkshops
\end{lstlisting}

\section{Additional Help}

Our regular academic semester office hours are every Tuesday and
Wednesday over zoom from 1--3:30 PM. See the green box on our home
page (\myurl{\docshomepage}) for the zoom link and
schedule details.

For documentation on creating Python environments, see 
\myurl{\mamba}. We are happy to demonstrate or help in office
hours.

\section{Extracurricular}

Interested in the student chapter of the Women in High-Performance
Computing (WHPC)? 
Please fill out the following interest form:
\myurl{\whpcform}.
Also subscribe to the club on SunDevil Sync, ASU's official centralized
resource for ASU clubs: 
\myurl{\whpcclub}.

\end{document}


